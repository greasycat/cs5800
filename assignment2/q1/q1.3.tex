\documentclass{article}
\author{Rongfei Jin}
\usepackage{amsmath}
\usepackage{amssymb}
\title{Assignment 2 - Question 1.3}
\date{\today}
\begin{document}
\section{Question 1.3}
\begin{align*}
    6^{22345} \mod 7
\end{align*}
\section{Solution}
Observe that there exists a pattern in the powers of 6 modulo 7:
\begin{align*}
    6^1 \mod 7 &= 6 \\
    6^2 \mod 7 &= 1 \\
    6^3 \mod 7 &= 6 \\
    6^4 \mod 7 &= 1 \\
    6^5 \mod 7 &= 6 \\
    6^6 \mod 7 &= 1
\end{align*}

Therefore, we suspect $6^{2k-1} \mod 7 = 6^{2 \times 11173 - 1} \mod 7 = 6,\forall k\in \mathbb{Z^+}$.
We can prove this with induction.

\subsection{Proof by induction}
Base case holds:
\begin{align*}
    6^{1} \mod 7 = 6^{6 \times 3724 + 1} \mod 7 = 6
\end{align*}

\noindent
Inductive step: assume when $n = k$, the statement $6^{2k-1} $holds, then consider $n=k+1$

\begin{align*}
    6^{2(k+1)-1} \mod 7 &= 6^{2k+1} \mod 7 \\
    &= 6^{2k-1} \times 6^2 \mod 7 \\
    &= ((6^{2k-1} \mod 7) \times (6^2 \mod 7)) \mod 7\\
    &= 6 \times 1 \mod 7 \\
    &= 6
\end{align*}

So the statement holds for $n = k+1 \forall k \in \mathbb{Z^+}$, and by the principle of mathematical induction, $6^{22345} \mod 7 = 6$.

\end{document}