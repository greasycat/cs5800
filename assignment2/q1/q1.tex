\documentclass{article}
\author{Rongfei Jin}
\usepackage{amsmath}
\usepackage{amssymb}
\title{Assignment 2 - Question 1.1}
\date{\today}


\begin{document}
\section{Question 1.1}
\begin{align*}
3^{1500} \mod 11
\end{align*}
\section{Solution}

\begin{align*}
3 \mod 11 &= 3 \\
3^2 \mod 11 &= 9 \\
3^3 \mod 11 &= 5 \\
3^4 \mod 11 &= 4 \\
3^5 \mod 11 &= 1 \\
3^6 \mod 11 &= 3 \\
3^7 \mod 11 &= 9 \\
3^8 \mod 11 &= 5 \\
3^9 \mod 11 &= 4 \\
3^{10} \mod 11 &= 1
\end{align*}

\noindent
Therefore, we suspect $3^{1500} \mod 11 = 3^{5 \times 500} \mod 11 = 1$.
we can prove this with induction.

\subsection{Proof by induction}
\noindent
To prove $$3^{5n} \mod 11 = 1$$

\noindent
Base case holds:
\begin{align*}
3^{1500} \mod 11 = 3^{10 \times 150} \mod 11 = 1
\end{align*}

\noindent
Inductive step: assume when n = k, the statement holds, then consider $n=k+1$
\begin{align*}
3^{5(k+1)} \mod 11 &= 3^{5k + 5} \mod 11 \\
&= 3^{5k} \times 3^5 \mod 11 \\
&= ( 3^{5k} \mod 11 \times 3^5 \mod 11 ) \mod 11 \\
&= 1 \times 1 \mod 11 \\
&= 1
\end{align*}
So the statement holds for $n = k+1\forall k \in \mathbb{Z^+}$, and by the principle of mathematical induction, $3^{5n} \mod 11 = 1$.

\subsection{Alternative Solution}
We can see that 
\begin{align*}
    3^{1500} \mod 11 &= 3^{4+8+16+64+128+256+1024} \mod 11 \\
    &= 3^4 \times 3^8 \times 3^{16} \times 3^{64} \times 3^{128} \times 3^{256} \times 3^{1024} \mod 11 \\
    &= (3^4 \mod 11 \times 3^8 \mod 11 \times \cdots \times 3^{1024} \mod 11 )\mod 11 \\
    &= 1
\end{align*}
\end{document}